%__________________________________________________________________________________________ %
%-------------------------------------------PFC---------------------------------------------%
%__________________________________________________________________________________________ %


%%%%%%%%%%%%%%%%%%%%%%%%%%%%%%  PREAMBULO  %%%%%%%%%%%%%%%%%%%%%%%%%%%%%%%%%%%%%
% ----------------------Especificaciones de dise�o---------------------------- %

\NeedsTeXFormat{LaTeX2e}
\documentclass[12pt]{book}

\usepackage{a4}
\usepackage[Lenny]{fncychap}    % Estilos para capitulos
\usepackage{fancyhdr}           % Estilos para cabeceras
%\usepackage[spanish]{babel}
%usepackage[latin1]{inputenc}

% Para no tener problemas con las tildes
\usepackage[utf8]{inputenc}
\usepackage[spanish]{babel}

\usepackage{epsfig}
\usepackage{subfig}
\usepackage{epstopdf}
\usepackage{caption}
\usepackage{keyval}
\usepackage{graphicx}
\usepackage{float}              % Para poner las imags en cualquier sitio
\usepackage{listings}
\usepackage{color}
\usepackage{textcomp}
\usepackage{verbatim}
\usepackage{exceltex}



\definecolor{listinggray}{gray}{0.98}
\definecolor{lbcolor}{rgb}{0.98,0.98,0.98}
\lstset{
	backgroundcolor=\color{lbcolor},
	tabsize=3,
	rulecolor=,
	language=matlab,
	basicstyle=\footnotesize\sffamily,
	aboveskip={1.5\baselineskip},
	belowskip={1.5\baselineskip},
	columns=fixed,
	showstringspaces=false,
	extendedchars=true,
	breaklines=true,
	prebreak = \raisebox{5ex}[5ex][5ex]{\ensuremath{\hookleftarrow}},
	frame=none,
	showtabs=false,
	showspaces=false,
	showstringspaces=false,
	identifierstyle=\ttfamily,
	keywordstyle=\color[rgb]{0,0,1},
	commentstyle=\color[rgb]{0.133,0.545,0.133},
	stringstyle=\color[rgb]{0.627,0.126,0.941},
}

\usepackage[numbers,sort&compress,comma]{natbib}	% Modo de poner la bibliograf�a
\usepackage{verbatim}      %  \begin{comment}...\end{comment}
\usepackage{subeqnarray}   % equationarray with numbers 1a, 1b, ...
\usepackage{bbm}           % Para s�mbolo tipo n�meros reales. Ej: \bbm{R}
\usepackage{longtable}     % Para tablas largas de m�s de una p�gina
\usepackage{rotating}
\usepackage{psfrag}        % Para cambiar fragmentos de text en .eps por otro en latex
\usepackage{pifont}        % Para otros s�mbolos
\usepackage{fancybox}      % Para encuadrar texto en recuadros
\usepackage{amsmath}       % Mejora la calidad de las formulas
\usepackage{amsfonts}
\usepackage [linktocpage]{hyperref}      % Para enlaces de hipertexto
\usepackage{amssymb,amsfonts}
\usepackage{multirow}
\usepackage{booktabs}
\usepackage{color}
\usepackage{longtable}
\usepackage{float}
\usepackage{array}




\setcounter{tocdepth}{2}             % toc = table of contents. Para definir niveles del �ndice
\setcounter{secnumdepth}{5}          % Hasta cu�ndo se enumeran los caps, seccs, etc


\setlength{\topmargin}{-1.1cm}        % margen por arriba
\setlength{\parskip}{\baselineskip}
\setlength{\parskip}{0.3cm}          % Espacio entre parrafos
\setlength{\textwidth}{16.5cm}       % Ancho del �rea imprimible	
\setlength{\evensidemargin}{-0.4cm}  % Margen izdo en p�ginas pares
\setlength{\oddsidemargin}{0.3cm}    % Margen izdo en p�ginas impares
% evensidemargin = -oddsidemargin !!!

\setlength{\headsep}{1.0cm}
\setlength{\headheight}{3ex}
\setlength{\footnotesep}{5mm}
%\setlength{\mathindent}{1.0cm}       % Controla el espacio entre margen y ec si no est� centrada


%%% Definitionen f�r Fancy Headings
%\renewcommand{\baselinestretch}{3mm}
%\renewcommand{\labelenumi}{\roman{enumi}.}
%\renewcommand{\chaptermark}[1]{\markboth{#1}{}}
%\renewcommand{\sectionmark}[1]{\markright{\thesection\ #1}{}}
\renewcommand{\labelitemi}{$\bullet$}
\renewcommand{\labelitemii}{$\diamond$}
\renewcommand{\labelitemiii}{$\cdot$}

\lhead[\fancyplain{}{\thepage}]{\fancyplain{}{\sl\nouppercase\rightmark}}
\rhead[\fancyplain{}{\sl\nouppercase\leftmark}]{\fancyplain{}{\thepage}}
\cfoot{}
\pagestyle{fancyplain}  		% normale Kopfzeile; ohne Seitenzahl: empty

% Formato de capitulos
\ChTitleVar{\sf\Huge} % Tama�o de la letra del nombre del cap
\ChTitleAsIs


%%% Comando para quitar encabezado y pie de las pag en blanco
\newcommand{\clearemptydoublepage}
{\newpage{\pagestyle{empty}\cleardoublepage}}

\newcommand{\R}{\mathbb{R}}
\newcommand{\x}{\mathbf{x}}
\newcommand{\grad}{\hspace{-2mm}$\phantom{a}^{\circ}$} %para los grados centigrados

%%% Abstract
\newenvironment{abstract}
{\begin{center}
		\begin{minipage}{0.8\textwidth}
			\slshape}
		{\end{minipage}
	\end{center}}
	
	
	\typeout{ }
	\typeout{----------------------------------------------------------------------}
	\typeout{ }
	
	



%%%%%%%%%%%%%%%%%%%%%%%%%%%%%%  DOCUMENTO  %%%%%%%%%%%%%%%%%%%%%%%%%%%%%%%%%%%%%
%-------------------------Cuerpo del documento---------------------------------%


\begin{document}

	\pagenumbering{roman}    % Numeraci�n de p�ginas con num romanos
	\setcounter{page}{1}      % Establece la siguiente p�gina como la 1
	\begin{titlepage}
\label{ch:portada}
\begin{center}

{\Large\textsc{Ingenier�a de Telecomunicaci�n}}


Departamentos:  \\ Arquitectura y Tecnolog�a de Computadores \\ Teor�a de la Se�al, Telem�tica y Comunicaciones

\textbf{Universidad de Granada}


\vspace{0.5cm}

\begin{figure}[h]
	\centering
	\epsfig{file=figuras/Portada/portada1, width=7cm}
	\label{fig:ugr}
\end{figure}


\vspace{0.5cm}
\textbf{PROYECTO FIN DE CARRERA}


\vspace{0.9cm}


{\Huge\textbf{Desarrollo de una aplicaci�n de visualizaci�n de datos y configuraci�n para sistemas de monitorizaci�n y an�lisis del movimiento del cuerpo humano.}}


\end{center}


\vspace{1.5cm}
\textbf{Realizado por:}  \hfill \textbf{Dirigido por:}

Javier L�pez Garc�a \hfill D. Alberto Olivares Vicente

\hfill D. Gonzalo Olivares Ruiz

\end{titlepage}
		% Inclu�mos la portada en espa�ol
	\clearemptydoublepage
	
	%Declaración
%--------

\begin{titlepage}
\label{ch:Statement}
\vspace{2cm}

\noindent  D. Alberto Olivares Vicente, profesor  del dpto. de Teoría de la Señal, Telemática y Comunicaciones, como director del Proyecto Fin de Carrera de Dª. Verónica Torres Sánchez,

\vspace{2cm}
\noindent Informan:

\vspace{1.5cm}
\noindent Que el presente trabajo, titulado:

\noindent \textbf{Comparison of Posturographic Body-sway Measurements with Inertial Data of Parkinson Patients.}

\noindent Ha sido realizado y redactado por el mencionado alumno bajo nuestra dirección, y con esta fecha autorizamos a su presentación.
\vspace{3.5cm}

\noindent Granada, a 20 de julio de 2015 Fdo:

\vspace{6.5cm}
\noindent D. Alberto Olivares Vicente    \hfill   D. Juan Manuel Górriz Sáez

\end{titlepage}       % Incluimos la declaración del proyecto
	\clearemptydoublepage
	\include{acknowlwdgements}      % Incluimos agradecimientos
	\clearemptydoublepage
	\thispagestyle{plain}
  \null\vfil
  
\begin{center}
    \setlength{\parskip}{0pt}
    {\normalsize \textsc{Münster University of Applied Sciences} \par}
    \bigskip
    {\huge{\textsc{Abstract}} \par}
    \bigskip
    {\normalsize Department of Electrical Engineering and Computer Science \par}
    \bigskip
    {\normalsize Bachelor of Science \par}
    \bigskip
    {\normalsize\bf Analysis of anticipatory postural adjustments of Parkinson's patients using inertial sensors \par} % Thesis title
    \medskip
    {\normalsize by Robin Weiß \par} % Author name
    \bigskip
  \end{center}

\noindent  
This thesis entitled ``Analysis of anticipatory postural adjustments of Parkinson's patients using inertial sensors'' was submitted in partial fulfilment of the requirements for the degree of Bachelor of Science in Electrical Engineering. 

I took part in a conjoint project between the Research Centre for Information and Communications Technologies of the University of Granada (CITIC-UGR) and the Department of Neurology of the Klinikum Großhadern in Munich, which is part of the Ludwig-Maximilians University. The objective of this thesis was to carry out an analysis of the so called anticipatory postural adjustments, which are the movements by a human subject between the moment he initiates gait and the first step.


      % Incluimos resumen
	\clearemptydoublepage
	
\begin{titlepage}
\label{ch:abbrevations}
{ \huge \bfseries Abbrevations \\[0.4cm] }

\textbf{APA}: Anticipatory postural adjustments

\textbf{SIPBA}: Signal processing and Biomedical Applications

\textbf{FP}: force plate

\textbf{GW}: Gait Watch

\textbf{QS}: Qualysis System

\textbf{PD}: Parkinson’s disease

\textbf{IMU}: Inertial Measurement Unit

\textbf{MIMU}: Magnetic Inertial Measurement Unit

\textbf{EMG}: Electromyography

\textbf{MEMS}: Microelectromechanical Systems

\textbf{LTSD}: Long Term Spectral Detector

\textbf{FSD}: Framed Spectrum Detector

\textbf{COP}: center of pressure

\textbf{AP}: Antero-posterior

\textbf{ML}: Medio Lateral

\textbf{FIR}: Finite Impulse Response.


\end{titlepage}       % Incluimos abreviaciones
	\clearemptydoublepage
	
	\tableofcontents          % Pone índice
	\clearemptydoublepage
	\renewcommand{\listfigurename}{List of figures}
	\renewcommand{\listtablename}{List of tables}
	

	\clearemptydoublepage
	\include{Introduccion}

	
	\clearemptydoublepage
	\bibliographystyle{unsrt}
	\bibliography{biblio}
	
	\clearemptydoublepage
\end{document}