\chapter{Conclusions and Future Work}
\label{ch:Conclusions}
\section{Conclusion}
The analysis of posture and human body motion  has gained special relevance within the past years in many medicine fields, as well as in others kinds of applications. Traditionally, this analysis has been carried out by Force Plates and systems base on cameras of high speed using infrared markers. This fact has restricted the fields of applications due to the price, portability and limitations of visibility. For this reason, we have carried out a comparative study of these systems in contrast with low cost devices based on magnetic and inertial MEMS microsensors (Gait Watch).

To do that, we have applied different techniques and procedures that we have explained and compared along the project.
In addition, our work has a direct application, not only in diagnosis of Parkinson’s disease, but also in neurological and muscular diseases such as Multiple Sclerosis, Cerebral Palsy and Epilepsy. In the application field, we focus on the analysis of posturographic  Body-sway to detect disorders in Parkinson patients as well as falls in elderly people which are becoming increasingly frequent.

Finally, we proceed to analyze the initial objectives that were enumerated in the introductory chapter to determine to what degree they have been fulfilled.


\begin{itemize}
\item \textbf{Synchronisation}

-	One of the objectives in this section is to carry out the synchronisation between FP and GW signals. We have several signals from both systems: force, AP-COP, ML-COP, etc. in the FP system and acceleration, angular velocity, pitch, roll, etc. in the GW system, having both systems a different sample frequency. Thus, the procedure is: choose the appropriate signals to do the synchronisation, detect a common point between them and interpolate them.
After a deep analysis of these signals, we concluded that the best signal to do the synchronisation is the acceleration of the shanks and the force over the platform because you can detect in both signals when the patient touches the platform and finally do a appropriate matching.
In addition, we compare the synchronisation done with the acceleration and angular velocity of the shanks obtaining a similar result that indicates this a good procedure to do the synchronisation.

-	Other goal emerged during the development of the project is the comparison between FSD (Framed Spectrum Detector) and LTSD (Long Term Spectral Detector) algorithm to determine the intensity motion. We determined after doing the comparison that LTSD is a better method for our signals because it is designed to work under condition where the SNR is low, i.e the signal presents large noise. In our case, we want to detect the different cycles in the GW signals that correspond to each repetition so the different peaks of activity inside each period can be a problem to do the detection correctly because they are interpreted as noise by the detector. Thus, the LTSD method is more interesting for this type of signal.


\item \textbf{APA analysis}

-	One of the main goals of this project is to analyse the APAs and determine whether there is a pattern that allows us to characterise the movement before gait. The signals used for this were the acceleration and angular velocity of the trunk and the center of pressure obtained from the force measures of the force plate. The conclusion is that we can see a pattern in all these signals and also it can be detected when the patient starts to step with the right or left foot. This is a great advance because we can obtain information and extract features of these signals.

-	Analysing the state of the art we can find that there are several articles that show what and how the features are extracted. Typically, the features more used to characterise this kind of movement are the peak of acceleration and COP. We used these features  as well as other peaks in these signals that we considered that could be interesting. Also, we calculated the duration of the APA in COP, acceleration signals and gyroscope signals.

-	We focus on the PCA method to extract features because it allows us the reduction of redundant information and the interpretation of multiple gait signals. We extract the same features mentioned above before and after applying PCA. We conclude that the acceleration and angular velocity can be substituted by only one of them because there is not variance between them. This is an important conclusion when we have a big data base. 
Also, we used PCA between patients, being this the first step to do a classification in a future work.

-	If we focus in chapter\ref{ch:GWandFP}, we can determine that there is a significant correlation of some of the features calculated between FP and GW signals. Although  we should expand the data base before drawing a definitive conclusion, we could say it is possible that devices based on inertial sensors can replace platforms. This would allow to do a lot of different experiments without limitations of space and price.

\item \textbf{Classification of force data}

One of the main goals in this experiment is to determine whether it is possible to classify Parkinson patients and healthy subjects from force data. We tested two different methods to extract features: PCA (unsupervised) and PLS (supervised) using SVM as classifier. The conclusion of this is: firstly, PCA is more appropriate from an accuracy viewpoint and PLS has better results than PCA from a sensitivity viewpoint, secondly we can obtain a good classification with both methods. This allows us to conclude that the analysis could be used to help in the diagnostic of Parkinson's disease.

\item \textbf{Qualisys system and Gait Watch (treadmill experiment)}

-	An initial objective was to determine what kinds of features we could extract in this type of experiment. The signals used to analyse this behaviour are the pitch angle of shanks and thighs. Thus, we concluded that the most interesting aspects to consider were ‘stride time’, variance of ‘stride time’  and peaks of angles. ‘Stride time’ has been a feature used with success in other similar studies so we though that it could be a significant parameter.

-	Determining the accuracy of the pitch angle obtained with GW and QS systems was another of our targets. To do this, we compare the features calculated from the pitch. The conclusion is that we can use the GW in place of QS in the majority of the cases if we need accuracy in the angle and ‘stride time’ because the difference of both is minimal.

-	In addition, we compare both systems in different conditions of speed to determine if the velocity affects in the accuracy of the systems. After the study, we can say that speed influences in the variance of ‘stride time’ and angle, i.e when the speed is higher the difference between systems to calculate the ‘stride time’ and angle increases. 
\end{itemize}

\section{Future Work}
As we have done for the final conclusions, we proceed to analyze future work for the two main parts of our system.

\begin{itemize}
\item \textbf{Force Plate and Gait Watch (experiment with Parkinson patients over a platform)}

- We will increase the data base of the patient for achieving results more accurately. Currently, we have a limited data base with only five patients. This gives us a rough guidance of the information that we can obtain but it is not decisive because to reach a definitive conclusion we have to have at least twenty patients.

-	We will add control subjects to our data base. The fact of having patients of Parkinson’s disease as well as control subjects would allow to do a classification. This is very important because it would really help in medical diagnosis of this disease.
The classification can be achieved through a SVM which separates a given labeled training dataset with a hyperplane that is maximally distant from the two classes, being these classes the patients and control subjects (or patients with different levels of the disease). The objective is to build a function using training data that will correctly classify new examples.

- We will test others kinds of algorithms to obtain features of the gait data. One of them is ICA (Independent Component Analysis).  ICA is a statistical technique that represents multidimensional random vector as a linear combination of nongaussian random variables (independent component) that are as independent as possible. ICA is somewhat similar to PCA to extract features. However, it would be interesting to compare both ICA and PCA to determine if we can extract the same information using both methods. 

- In addition, all the work developed in this project could be used in other applications. For example in patients with motor disorders or elderly people and establish if this study can be a widespread procedure to others fields.

\item \textbf{Classification of force data}

The classification has been done with force data while subjects (both patients and control subjects) were walking normally. We will try to do the classification with other types of experiments where subjects are carrying out different activities. It would allow us to extend the study to whatever daily activity.

\item \textbf{Qualisys system and Gait Watch (treadmill experiment)}

-	As in the first case, we will increase the data base. This not only allows to have more accurate results but also we can do others types of studies to deepen the knowledge of the Posturographic Body-sway  with inertial data.

-	We will analyse the movements not only of shanks and thighs but also of trunk and arms carrying out other experiments with Qualisys System and Gait Watch.
	
\end{itemize}