\begin{titlepage}
\label{ch:abstract}
{ \huge \bfseries Abstract \\[0.4cm] }

Posturographic Body-Sway Measurements are increasingly important in many fields of modern medicine, sport activities and teleassistance.  Postural instability is an important contributor to incapacitation in elderly people and patients with neurological or motion disorders like people with Parkinson’s disease.

This work describes the development  of a comparative study between several systems to measure the Posturographic Body-sway and techniques and procedures to extract features of them. Along this document we describe, firstly, the most common  instrumentation in these kind of experiments as well as the methods  used for data analysis of Anticipatory Postural Adjustment (APA) in Parkison patients. Secondly, we will introduce a general description of all devices used to gather the necessary data  for the development of  this project. These devices are: the Gait Watch, a system based on inertial sensors for gait monitoring, a Force Platform, a system for force measurements and the Qualisys System, a system that uses speed digital cameras and markers attached to track motion.

The core of the project consists of comparing the signals obtained from the systems mentioned above. In the first place, we are going to compare the GW anf FP signals to determine whether we can obtain the same information of both systems and we can use the inertial sensors in place of force platforms. To do this, we will carry out the synchronisation of the signals, analysis of APA and the feature extraction of these signals. In this last aspect, we focus in PCA algorithm because it allows us the reduction of redundant information and the interpretation of multiple gait signals. In second place, we are going to analyse the Gait Watch system in comparison with the Qualisys Optical motion tracker. In this case, we will calculate the pitch angle for both systems. We will compare features like ‘stride time’ and angle. Before doing this, we explain  the configuration of the  Qualisys System and how we compute the angle for each point of time. In addition, we will do a feature extraction using both PCA and PLS algorithms and classification of data force in another different experiment with Parkinson patients and healthy people. This allows us to determine the accuracy of force data for the diagnosis of this disease.

Finally, we present a summary of the different fields of applications related to health-care and a brief  business plan where we explain a description of a business idea and the most important aspects to take into account.

\end{titlepage} 