\chapter{Conclusion and Future Work}
\label{ch:Conclusion and Future Work}

\section{Conclusions}

Summarising the above, I can say that I have learned a lot in the four month that I spent in Granada. Amongst others I have come to know many new work methods, not only due to being exposed to people from a different culture, but also due to the fact that scientific research differs strongly from the work as a student at university. I gained a deeper understanding of Parkinson's disease and how various gait analysis techniques are used to quantify its effects. Therefore I had to study the principles of force plates and inertial measurement units as well as the basics of classification. I was able to improve my MATLAB$\textsuperscript{\textregistered}$ skills and have realised how important it is to write understandable and well commented code, if it is for a larger project and not only for a coursework. I am now familiar with tools such as GitHub and Pivotal Tracker which make working in a team much easier and significantly more efficient.  Beside my work at the research centre, where I obtained a valuable insight into scientific research, I read a book about scientific writing that helped me to improve my oral and written English skills during my stay. Furthermore I now know the fundamentals of \LaTeX{}.

The above will hopefully serve as a good foundation for my subsequent bachelor's thesis.  All in all it was a great experience, professionally as well as personally. I truly recommend such a stay to every university student.

\section{Future Work}

Biomedical research is a very interesting blend of both my major interests, that is, working in the medical field as a paramedic and  in the technical field as an electrical engineer. I would like to keep working in this field and write my aforementioned bachelor's thesis here in Granada. There is a variety of possible future work. One related topic would be the validation of the pitch angles of the body measured with the GaitWatch by means of cameras that record the trace of visual markers. From these markers one could compute the pitch angels and compare them to those of the GaitWatch to validate its performance. Another option would be the implementation of an new Kalman filter based orientation algorithm proposed by \cite{} to process the information of the GaitWatch. Also a visual interface for the entire gait analysis process would be a reasonable improvement of the existing system.