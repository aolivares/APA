\chapter{Introduction}
\label{ch:Introduction}

\section{General}

\subsection{Parkinson's Disease}

According to Patients Medical \cite{patients_medical_definition_2014}, \begin{quote}``Parkinson's disease is a progressive, neurodegenerative disease that occurs when the neurons within the brain responsible for producing the chemical dopamine become impaired or die. Dopamine is essential for the smooth control and coordination of the movement of voluntary muscle groups. Once approximately 80\% of the brain's dopamine producing cells no longer function, the symptoms of Parkinson's disease begin to appear. \dots Parkinson's disease may be termed as a progressive movement disorder that is distinguished by marked slow movements, tremors, and unstable posture.''\end{quote}

Especially in advanced stages of the Parkinson's disease (PD)\nomenclature{PD}{Parkinson's disease} many patients exhibit an episodic, brief inability to step that delays gait initiation or interrupts ongoing gait. This phenomenon is called freezing of gait and is often associated with an alternating shaking of the knees, termed as knee trembling. However, these clinical signs of balance or gait problems are not evident in early stages of the disease \cite{mancini_anticipatory_2009}\cite{jacobs_knee_2009}.

\subsection{Anticipatory Postural Adjustments}

To induce gait Anticipatory postural adjustments (APAs)\nomenclature{APAs}{Anticipatory Postural Adjustments} are made. The Encyclopedia of Neuroscience \cite[p.133]{woollacott_anticipatory_2009} defines APAs as "A predictive motor response that acts to counter, in a preemptive manner, the postural destabilization associated with a forthcoming movement." The centre of body mass (COM)\nomenclature{COM}{Centre of Mass} is accelerated forward and laterally over the stance foot to make sure that the body does not fall laterally toward the stepping foot during the swing phase \cite{woollacott_anticipatory_2009}. Therefore the centre of pressure (COP)\nomenclature{COP}{Centre of Pressure} is moved posteriorly and toward the stepping leg \cite{mancini_anticipatory_2009}. Figure \ref{fig:APAoverview} shows the whole process divided in three periods. Period S1 indicates the uncoupling of the COP and COM as the COP moves posteriorly and toward the intended stepping limb. Subsequently, in the S2 period, the COP displaces mediolaterally toward the stance foot. Finally, during the S3 period the COP moves anteriorly under the stance foot \cite{hass_gait_2005-1}.

\begin{figure}
	\centering
	\epsfig{file=images/APA_overview, width=9cm}
	\caption{Anticipatory Postural Adjustments during forward-oriented gait initiation when stepping with the right foot. The arrow represents the distance between the COP-COM \cite{hass_gait_2005-1}}
	\label{fig:APAoverview}
\end{figure}


\section{Motivation}

Advanced PD can increasingly diminish quality of life due to the fact that patients are dependent on help from other doing daily tasks. Early diagnosis of PD could 

optimise treatment

assess treatment success

ambulant long term measurements


\section{Goals}

The main goal was to build a classifier which is fed with data from both force plate and magnetic inertial measurement unit (MIMU)\nomenclature{MIMU}{Magnetic Intertial Measurement Unit} to distunguish between Parkinson patients and healthy subjects.


\section{State of the art}

There are several methods and devices to assess Parkinson's disease and to analyse Anticipatory Postural Adjustments. The state of the art at the beginning of the project is presented below.

\subsection{Rating scales}

One commonly used rating scale is the Unified Parkinson’s Disease Rating Scale (UPDRS)\nomenclature{UPDRS}{Unified Parkinson’s Disease Rating Scale} which is a short test performed by a physician \cite{klerk_long-term_2009}. Another method is the widely utilised and accepted Hoehn and Yahr scale (HY)\nomenclature{HY}{Hoehn and Yahr scale}. Parkinsonian motor impairment is split in 5 stages: Unilateral (Stage 1) to bilateral disease (Stage 2) without balance difficulties, to the presence of postural instability (Stage 3), loss of physical independence (Stage 4), up to being wheelchair- or bed-bound (Stage 5) \cite{goetz_movement_2004}. \citeauthor{klerk_long-term_2009} \cite{klerk_long-term_2009} mentioned the disadvantages such as subjectivity, short observation periods and unfamiliarity of the environment that both rating methods bring along.

\subsection{Instrumentation}

In addition to the named subjective rating scales there are different devices used to quantify gait and posture and assess them ojectively. All of them come with certain pros and cons. The following devices have been used:

\begin{itemize}

\item Electromyographs:

\item Force plates:

\item Inertial sensors:

\item Camera based:

\end{itemize}

\subsection{Calibration}

Different calibration methods are used...


\subsection{Classification}

Different classification methods/techniques are used...

Signal processing

