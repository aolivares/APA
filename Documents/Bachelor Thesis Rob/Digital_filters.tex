\chapter{Digital Filters}
\label{ch:digital_filters}

Conceived in general terms, a filter is a physical device for removing unwanted components of a mixture. In the technical field a filter is a system designed to extract information from noisy, distorted data. That is, the filter delivers an estimate of the variables of principal interest, which is why it may also be called an estimator. Filter theory is applied in diverse fields of science and technology, such as communications, radar, sonar, navigation, and biomedical engineering \cite{haykin2002adaptive}.

In contrast to analogue filters that consist of electronic circuits to attenuate unwanted frequencies in continuous-time signals and thus extracted the useful signal, a digital filter is a set of mathematical operations applied to a discrete-time signal in order to extract information about the hidden quantity of interest. A discrete-time signal is a sequence of samples at equidistant time instants that represent the continuous-time signal with no loss, provided the sampling theorem is satisfied, according to which the sample frequency has to be greater than twice the highest frequency component of the continuous-time signal.

Digital filters can be classified as linear and nonlinear. If the quantity at the output of the filter is a linear function of its input, that is, the filter function satisfies the superposition principle, the filter is said to be linear. Otherwise, the filter is nonlinear.

\section{The Filtering Problem}

Consider, as an example involving filter theory, the continuous-time dynamical system depicted in Figure \ref{fig:state_estimation}. The desired state vector of the system, \gls{not:x(t)_v}, is usually hidden and can only be observed by indirect measurements \gls{not:y(t)_v} that are a function of \gls{not:x(t)_v} and subject to noise. Equally, the equation describing the evolution of the state \gls{not:x(t)_v} is usually subject to system errors. The dynamical system may be an aircraft in flight, in which case the elements of the state vector are constituted by its position and velocity. The measuring system may be a tracking radar producing the observation vector \gls{not:y(t)_v} over an interval $[0, T]$. The requirement is to deliver a reliable estimate \gls{not:x_hat(t)_v} of the actual state \gls{not:x(t)_v}, taking prior information into account.

\tikzstyle{block} = [draw, rectangle, minimum height=3em, minimum width=6em]
\tikzstyle{output} = [coordinate]
\tikzstyle{pinstyle} = [pin edge={to-, thin, black}, align=center]

\begin{figure}
\centering
\begin{tikzpicture}[auto, node distance=3cm,>=latex']
    \node [block, align=center, 
    	pin={[pinstyle]below:System \\ Errors}]
    	(dynamical) {Dynamical \\ system};
    \node [block, align=center, right of=dynamical, pin={[pinstyle]below:Measurement \\ errors}, node distance=4.5cm] (measuring) {Measuring \\ system};
    \node [block, align=center, right of=measuring, pin={[pinstyle]below:Prior \\ information}, node distance=4.5cm] (estimator) {Estimator};
    \node [output, right of=estimator] (output) {};
    
    \draw [->, align=center] (dynamical) -- node[name=x] {State \\ \gls{not:x(t)_v}} (measuring);
    \draw [->, align=center] (measuring) -- node[name=y] {Observation \\ \gls{not:y(t)_v}} (estimator);
    \draw [->, align=center] (estimator) -- node[name=y] {Estimate \\ of state \\ $\hat{\mathbf{x}}(t)$} (output);
\end{tikzpicture}
\caption{Block diagram depicting the components involved in state estimation adopted from \cite{haykin2002adaptive}.} \label{fig:state_estimation}
\end{figure}

\section{Wiener Filter}

A statistical criterion, according to which the performance of a filter can be measured, is the mean-squared error. Assuming a stationary stochastic process with known statistical parameters as the mean and correlation functions of the useful signal and the unwanted additive noise, the so-called Wiener filter is said to be optimum as it minimises the mean-square value of the error signal. This error signal is defined as the difference between some desired response and the actual filter output.

According to \citeauthor{haykin2002adaptive} \cite{haykin2002adaptive} the essence of the filtering problem is summarised with the following statement:


Since the Wiener filter requires a priori information about the statistics of the data to be processed, it may not be optimum for non-stationary processes.



\section{Adaptive Filters}

A possible approach to mitigate the limitations of the Wiener filter for non-stationary processes is the `estimate and plug' procedure. The filter `estimates' the statistical parameters of the relevant signals and `plugs' them into a non-recursive formula for computing the filter parameters. This procedure requires excessively elaborate and costly hardware for real-time operation \cite{haykin2002adaptive}. To overcome this disadvantage one may use an adaptive filter, which is a self-designing system that relies, in contrast, on a recursive algorithm. It allows the filter to perform satisfactorily even if there is no complete knowledge of the relevant signal characteristics. Provided the variations in the  statistics of the input data are sufficiently slow, the algorithm can track time variations and is thus suitable for non-stationary environments. The algorithm starts from some predetermined set of initial conditions respecting the knowledge about the system. In a stationary environment it converges to the optimum Wiener solution in some statistical sense after successive iterations.

Due to the fact that the parameters of an adaptive filter are updated each iteration, they become data dependent. The system does not obey the principles of superposition which therefore makes the adaptive filter in reality a nonlinear system. However, an adaptive filter is commonly said to be linear if its input-output map satisfies the superposition principle, as long as its parameters are held fixed. Otherwise it is said to be nonlinear.


\section{Kalman Filter}

The following simple example from \cite{Maybeck79} is no complete mathematical derivation but an illustrative description of the determination of a one-dimensional position to understand how the Kalman filter works.

Suppose you are lost at sea during the night and take a star sighting to determine your approximate position at time $t_1$ to be $z_1$. Your location estimate is, due to inherent measurement device inaccuracies and human error, somewhat uncertain, and thus assumed to be associated with a standard deviation is $\sigma_{z_1}$. The conditional probability of $x(t_1)$, your actual position at time $t_1$, conditioned on the observed value $z_1$, is depicted in Figure 

\begin{tikzpicture}

\draw[thick] plot[samples=200, smooth, domain=-.5:5.5] (\x, {0.1+2.8/(0.9*sqrt(2*pi))*exp(-((\x-2.5)^2)/(2*0.9^2))}) coordinate[pos=0.8] (A);
  \draw[->] (-1,0) -- (6,0) node[right] {$x$};
  \draw[->] (0,-1) -- (0,2) node[right] {$f_{x(t_1)|z(t_1)}(x|z_1)$};
  
  
\end{tikzpicture}

\section{Extended Kalman Filter}


