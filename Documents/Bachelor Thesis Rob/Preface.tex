\chapter{Preface}

This thesis was submitted in partial fulfilment of the requirements for the degree of Bachelor of Science in Electrical Engineering. It describes the implementation of a new Kalman filter based orientation algorithm to improve the estimation of orientation angles by means of inertial sensors.

I took part in the joint research project “Human Body Motion Analysis of Patients with Neurodegenerative Diseases by Means of Inertial Sensors” between the \gls{CITIC-UGR}, Spain, and the Department of Neurology of the Klinikum Großhadern, which is part of the Ludwig Maximilian University of Munich, Germany. The goal of the overall project was to obtain several gait parameters by wearable inertial sensors and validate them against conventional methods, such as cameras in combination with visual markers and force measuring platforms. Physicians and medical researchers are interested in this approach of body motion analysis, as unobtrusive wearable sensors can assist the diagnosis of neurodegenerative diseases, such as, for instance Parkinson's disease. Prior to this thesis I completed a three-months internship at the \gls{CITIC-UGR} in which I worked on the synchronisation of a force measuring platform with inertial sensors within the above-mentioned project.