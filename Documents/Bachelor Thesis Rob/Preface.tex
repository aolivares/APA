\chapter*{Preface}

This thesis was submitted in partial fulfilment of the requirements for the degree of Bachelor of Science in Electrical Engineering. It describes the implementation of a new Kalman filter based orientation algorithm to improve the estimation of the orientation angels obtained with a system called GaitWatch that is used for stance and gait analysis by means of inertial sensors.

I took part in the joint research project “Analysis of anticipatory postural adjustments of Parkinson’s patients using inertial sensors” between the \gls{CITIC-UGR} and the Department of Neurology of the Klinikum Großhadern in Munich, which is part of the Ludwig-Maximilians University. The goal of this project was to carry out an analysis of the so called anticipatory postural adjustments, which are the movements by a human subject between the moment he initiates gait and the first step. The medical community is interested in this procedure, as it can assist the diagnosis of neurodegenerative diseases such as Parkinson's. Prior to this thesis I completed a three-months internship at the \gls{CITIC-UGR} in which I worked on the synchronisation of a force measuring plate and the GaitWatch system within the above-mentioned project.