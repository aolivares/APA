\chapter{MARG Sensors}
\label{ch:MARG}

MARG sensors is a collective term for magnetic, angular rate, and gravitational sensors. It encompasses inertial sensors, such as accelerometers and gyroscopes, as well as magnetic field sensors, also referred to as magnetometers. This chapter compiles the functional principles of MARG sensors and introduces \glspl{IMU} as a combination of those. At the end of the chapter the GaitWatch device is described in detail.

\section{Accelerometers}

Accelerometers measure the acceleration of an object relative to an inertial frame. Since acceleration cannot be measured directly, the force exerted to a reference mass is obtained and the resultant acceleration is computed according to Newton's second law $\mathbf{F} = m \cdot \mathbf{a}$, where $ \mathbf{F}$ denotes the three-dimensional force vector, $m$ the mass, and $\mathbf{a}$ the three-dimensional acceleration vector. Microelectromechanical systems (MEMS) accelerometer 

\section{Gyroscopes}

Gyroscopes measure angular velocity and are based on the Coriolis Effect. By integrating the angular velocity the rotation angle is obtained \cite{olivares_vicente_signal_2013}.

\section{Magnetometers}

Magnetometers measure the strength and the direction of the magnetic field in a point in space, using the relationship between magnetic fields, movement and induced currents \cite{olivares_vicente_signal_2013}.

\section{Inertial Measurement Units}

Devices using a combination of accelerometers and gyroscopes to measure the orientation of a solid body with up to six degrees of freedom are referred to as \glspl{IMU}. If they include magnetic field sensors (magnetometers), they are termed \glspl{MIMU}.

\glspl{MIMU} are portable and relatively inexpensive. They can be easily attached to the body and thus allow non-clinical longterm application. Their drawbacks are complex calibration procedures and drift behaviour over time, depending on intensity and duration of the movement. Hence, in order to maintain a satisfactory degree of precision, periodical recomputation of the calibration parameters is required \cite{olivares_vicente_signal_2013}.

\section{The GaitWatch}

The above mentioned GaitWatch device we used to gather the movement data was designed to monitor the motion of patients while attached to the body. It was developed at the Department of Neurology of the Ludwig-Maximilians University in Munich, Germany, in association with the Department of Signal Theory, Telematics and Communications of the University of Granada, Spain. The system is composed of a set of embedded magnetic and inertial sensors wired to a box containing a microcontroller. This microcontroller is in charge of collecting data from the embedded box sensors, as well as from the external measurement units, and storing them on a memory card. The various units are placed at the patient's trunk, arms, thighs, and shanks as shown in Figure \ref{fig:GaitWatch_placement}. The components of the three different kinds of subunits are described below:


\begin{itemize}

\item \textsc{Type A} -- thighs and shanks: 

IMU Analog Combo Board with 5 Degrees of Freedom \cite{IMU5}, containing an IDG500 biaxial gyroscope, from which only y-axis is actually used, with a measurement range of ±500\,°/s \cite{IDG500} and a ±3\,g triaxial accelerometer, ADXL335 \cite{ADXL335}.

\item \textsc{Type B} -- arms:

IDG500 biaxial gyroscope with a measurement range of ±500\,°/s \cite{IDG500}.

\item \textsc{Type C} -- trunk:

ADXL345 triaxial accelerometer with a programmable measurement range of ±2/±4/±8/±16\,g \cite{ADXL345},
IMU3000 triaxial gyroscope with a programmable measurement range of ±250/±500/±1000/±3000°/s \cite{IMU3000}, 
Micromag3 \allowbreak triaxial magnetometer with a measurement range of ±11\,Gauss \cite{MicroMag3}, AL-XAVRB board containing an AVR ATxmega processor \cite{AVRATxmega}.

\end{itemize}

\begin{figure}
	\centering
	\epsfig{file=images/GaitWatch_placement, width=9cm}
	\caption{Placement of the GaitWatch components at the body from \cite{olivares_vicente_gaitwatch_2013}.}
	\label{fig:GaitWatch_placement}
\end{figure}
