\chapter{Implementation}
\label{ch:Implementation}

This chapter describes the theoretical design of the system and the implementation, based on the fundamentals acquired in the previous the chapters. Subsequently, the experiments and its results are presented.

\section{Initial Situation}

 (Any algorithms that are already implemented for comparison that I can cite or state and describe how they differ from the one that I will implement??)

\section{Theoretical Design}

According to \citeauthor{bennett_motion_2014} \cite{bennett_motion_2014}, when walking in a straight line, the human leg can be modelled as a two-link planar revolute robot. As depicted in Figure \ref{fig:robot}, the revolute joints represent the hip und knee joint, and the two links the thigh and shank, respectively.

\begin{figure}
\centering
\begin{tikzpicture}[auto, thick, node distance=3cm,>=latex']
    \node [draw, very thick, rectangle, rounded corners=3pt, minimum height=3.5cm, minimum width=1cm, align=center, rotate around={30:(0,0)}] at (0.7, -1.25) (link1) {};
    \node [draw, thick, fill=gray, rounded corners=2pt, rectangle, minimum height=0.8cm, minimum width=0.5cm, align=center, rotate around={30:(0, 0)}, label={[label distance=0.1cm]right:Sensor 1}] at (0.98, -1.7) (sensor1) {};
    
    \node [draw, fill=white, very thick, rectangle, rounded corners=3pt, minimum height=3.5cm, minimum width=0.6cm, align=center, rotate around={145:(0,0)}] at (0.6, -3.7) (link2) {};
    \node [draw, thick, fill=gray, rounded corners=2pt, rectangle, minimum height=0.8cm, minimum width=0.5cm, align=center, rotate around={145:(0, 0)}, label={left:Sensor 2}] at (0, -4.56) (sensor2) {};
    
    \node[coordinate] (X) at (2.5,0) {};
    \node[coordinate] (Y) at (0,2.5) {};
    \node[coordinate] (O) at (-0, 0) {};
    
    \draw[->] (O) -- node[name=x] {$x$}(X);
    \draw[->] (O) -- node[name=y] {$y$}(Y);
    \draw[dotted] (O) -- (2.5,-4.3);
    \draw[fill=white] (0,0) circle (4pt);
    \draw (1.45, -2.5) circle (4pt);
    
    \draw[-stealth] (1.2,-1) arc (300:355:1.2);
    \draw[-stealth] (0.8,-4.1) arc (240:303:1.4);
    
    \node at (1.3, -0.4) (angle1) {$\alpha_1$};
    \node at (1.5, -3.7) (angle1) {$\alpha_2$};
\end{tikzpicture}
\caption{Acceleration seen by the sensor (b) with and (a) without motion from \cite{bennett_motion_2014}.} \label{fig:robot}
\end{figure} 

\section{Implementation}

The filter algorithm was implemented in \textsc{Matlab}$\textsuperscript{\textregistered}$.

\section{Experiments}

\section{Results}

\section{Discussion}

