\chapter{Implementation}
\label{ch:Implementation}

This chapter describes the theoretical design of the system and the implementation, based on the fundamentals acquired in the previous the chapters. Subsequently, the experiments and its results are presented.

\section{Initial Situation}

 (Any algorithms that are already implemented for comparison that I can cite or state and describe how they differ from the one that I will implement??)

\section{Theoretical Design}

This section maps the theoretical design of the system proposed by \citeauthor{bennett_motion_2014} in \cite{bennett_motion_2014}, that is, the kinematic model and the extended Kalman filter, to the existing GaitWatch system and its associated conventions. 

\subsection{Kinematic model}

When walking in a straight line, the human leg can be modelled as a two-link planar revolute robot \cite{bennett_motion_2014}. As depicted in Figure \ref{fig:robot}, the revolute joints of the \gls{pendubot} represent the hip und knee joint, and the two links the thigh and shank, respectively. The origin of the inertial navigation frame is located at the base of link 1, the upper of both links. The angle $\psi_1$ is measured with respect to the $x$-axis, and the angle $\psi_2$ of link 2, with respect to link 1.

\begin{figure}
\centering
\begin{tikzpicture}[auto, thick,>=latex']
	\node [draw, rectangle, minimum height=1.6cm, minimum width=1.6cm, pattern=north west lines] at (0,0) (solid) {};

    \node [draw,  fill=white, very thick, rectangle, rounded corners=3pt, minimum height=3.5cm, minimum width=1cm, align=center, rotate around={30:(0,0)}] at (0.7, -1.25) (link1) {};
    \node [draw, thick, fill=gray, rounded corners=2pt, rectangle, minimum height=0.8cm, minimum width=0.5cm, align=center, rotate around={30:(0, 0)}, label={[label distance=0.18cm]335:IMU 1}] at (0.98, -1.7) (sensor1) {};
    
    \node [draw, fill=white, very thick, rectangle, rounded corners=3pt, minimum height=3.5cm, minimum width=0.6cm, align=center, rotate around={145:(0,0)}] at (0.6, -3.7) (link2) {};
    \node [draw, thick, fill=gray, rounded corners=2pt, rectangle, minimum height=0.8cm, minimum width=0.5cm, align=center, rotate around={145:(0, 0)}, label={left:IMU 2}] at (0, -4.56) (sensor2) {};
    
    \node[coordinate] (X) at (2.5,0) {};
    \node[coordinate] (Y) at (0,2.5) {};
    \node[coordinate] (O) at (-0, 0) {};
    
    \draw[->] (O) -- node[name=x] {$x$}(X);
    \draw[->] (O) -- node[name=y] {$y$}(Y);
    \draw[dotted] (O) -- (2.5,-4.3);
    \draw[fill=white] (0,0) circle (4pt);
    \draw (1.45, -2.5) circle (4pt);
    
    \draw[-stealth] (1.2,-1) arc (300:355:1.2);
    \draw[-stealth] (0.8,-4.1) arc (240:303:1.4);
    
    \node at (1.3, -0.4) (angle1) {$\psi_1$};
    \node at (1.5, -3.7) (angle1) {$\psi_2$};

\end{tikzpicture}
\caption{Acceleration seen by the sensor attached to the pendubot (b) with and (a) without motion from \cite{bennett_motion_2014}.} \label{fig:robot}
\end{figure} 

The \glspl{IMU} on the thighs and shanks measured the angular velocity and linear acceleration of the thighs and shanks, respectively. According to \citeauthor{spong2005robot} \cite{spong2005robot}, the $x$ and $y$ displacement and the related derivatives in the world frame are

\begin{align}
  x &= a_1 \cos \psi_1 + a_2 \cos(\psi_1 + \psi_2) \\
  \dot{x} &= -a_1 \dot{\psi}_1 \sin \psi_1  - a_2 (\dot{\psi}_1 + \dot{\psi}_2) \sin(\psi_1 + \psi_2) \\
  \ddot{x} {}&= -a_1 [\dot{\psi}^2_1 \cos \psi_1 + \ddot{\psi}_1 \sin \psi_1] - a_2 [(\dot{\psi}_1 + \dot{\psi}_2)^2 \cos(\psi_1 + \psi_2) \nonumber \\ 
  &\mathrel{\phantom{=}} + (\ddot{\psi}_1 + \ddot{\psi}_2) \sin(\psi_1 + \psi_2)] \label{eq:acc_x} \\
  \nonumber \\
  y &= a_1 \sin \psi_1 + a_2 \sin(\psi_1 + \psi_2) \\
  \dot{y} &= a_1 \dot{\psi}_1 \cos \psi_1  + a_2 (\dot{\psi}_1 + \dot{\psi}_2) \cos(\psi_1 + \psi_2) \\
  \ddot{y} {}&= a_1 [\ddot{\psi}_1 \cos \psi_1 - \dot{\psi}^2_1 \sin \psi_1] + a_2 [(\ddot{\psi}_1 + \ddot{\psi}_2) \cos(\psi_1 + \psi_2) \nonumber \\ 
  &\mathrel{\phantom{=}} - (\dot{\psi}_1 + \dot{\psi}_2)^2 \sin(\psi_1 + \psi_2)] \label{eq:acc_y}
\end{align}

\noindent
in which $a_1$ and $a_2$ are the lengths of the two links, respectively.

The orientation of the sensor frames at rest are different from the world frame and dynamic when the pendulum is in motion. To transform the values from the world frame to the dynamic body frame of \gls{IMU} 2, which is depicted in Figure, we used the transformation matrix $\mathbf{T}_z(\psi)$ in Equation \ref{eq:transformation_matrices} with $\psi = \psi_1 + \psi_2$, which yields

\begin{equation}
\mathbf{T}_z(\psi) = \begin{bmatrix}
    \cos (\psi_1 + \psi_2) \; & \sin (\psi_1 + \psi_2) \; & 0 \\
    -\sin (\psi_1 + \psi_2) \; & \cos (\psi_1 + \psi_2) \; & 0 \\
    0 \; & 0 \; & 1
    \end{bmatrix}\,.
\end{equation}

\noindent
The rotated tangential and radial components of the motion based acceleration estimates, $A_{tan}$ and $A_{rad}$ are found using $\mathbf{T}_z(\psi)$ to rotate the results of Equations \ref{eq:acc_x} and \ref{eq:acc_y}, according to Equation \ref{eq:transformation}, respectively.

Then, the tangential and radial acceleration estimates are subtracted from the sensor readings $A_x$ and $A_y$, which leaves an estimate of the gravity based acceleration $\mathbf{g}$ that acts on the sensor:

\begin{equation}
\begin{bmatrix}
    g_x \\
    g_y 
    \end{bmatrix} = 
    \begin{bmatrix}
    A_x \\
    A_y 
    \end{bmatrix} -
    \begin{bmatrix}
    A_{rad} \\
    A_{tan} 
    \end{bmatrix}\,.
\end{equation}

\noindent
According to Equation \ref{eq:projection_gravity} the improved angle estimate is

\begin{equation}
  \theta = \mbox{atan2}(\frac{g_y}{g_x})\,.
\end{equation}

\noindent
This can be used to reduce the estimation error due to gyroscope drift.

\subsection{Extended Kalman Filter Model}

The state space model of the extended Kalman filter is given by the state vector

\begin{equation} \label{eq:state_vector}
  \mathbf{x} = \begin{bmatrix}
  	x, & y, & \psi_1, & \omega_1, & \alpha_1, & \psi_2, & \omega_2, & \alpha_2, & \beta_1, & \beta_2
  \end{bmatrix}^T
\end{equation}

where $\psi_1$ is the angle, $\omega_1$ the angular velocity, and $\alpha_1$ the angular acceleration of the first joint, respectively. The corresponding values for the second link are $\psi_2$, $\omega_2$, and $\alpha_2$. The biases from the gyroscope on the first and second sensor are $\beta_1$ and $\beta_2$, respectively. They are assumed to be constant or slowly time varying.

The measurement matrix $\mathbf{y}$ is given by

\begin{equation} \label{eq:measurement_vector}
  \mathbf{y} = \begin{bmatrix}
  	\omega_1 + \beta_1, & \omega_1 + \omega_2 + \beta_2, & \psi_1 + \psi_2
  \end{bmatrix}^T\,.
\end{equation}
 
\noindent
The element $y_1$ represents the measurement of the first link angular velocity, which is the sum of the first link rotation and the gyroscope 1 bias. Equally, the element $y_2$ represents the measurement of the second link angular velocity, which is the sum of the first and second link rotation and the bias of gyroscope 2. Finally, the element $y_3$ is the angle estimate of the second accelerometer, which will see the angular displacement of both links.

(As in 20 )The linear approximation of the state equations at each iteration yields the state transition matrix

\begin{equation}
  \bm{\Phi}^{[1]}_{k} =  \mathbf{I}_{10} + \mathbf{F} T_s\,,
\end{equation}

\noindent
with

\begin{equation}
\mathbf{F} = \begin{bmatrix}
  0 & 0 & 0 & A & 0 & 0 & B & 0 & 0 & 0\\
  0 & 0 & 0 & C & 0 & 0 & D & 0 & 0 & 0\\
  0 & 0 & 0 & 1 & 0 & 0 & 0 & 0 & 0 & 0\\
  0 & 0 & 0 & 0 & 1 & 0 & 0 & 0 & 0 & 0\\
  0 & 0 & 0 & 0 & 0 & 0 & 0 & 0 & 0 & 0\\
  0 & 0 & 0 & 0 & 0 & 0 & 1 & 0 & 0 & 0\\
  0 & 0 & 0 & 0 & 0 & 0 & 0 & 1 & 0 & 0\\
  0 & 0 & 0 & 0 & 0 & 0 & 0 & 0 & 0 & 0\\
  0 & 0 & 0 & 0 & 0 & 0 & 0 & 0 & 0 & 0\\
  0 & 0 & 0 & 0 & 0 & 0 & 0 & 0 & 0 & 0\\
\end{bmatrix}\,,
\end{equation}

\noindent
and

\begin{equation*}
  \begin{array}{cc}
  \begin{split}
  	A &= -a_1 \sin \psi_1 -a_2 \sin (\psi_1 + \psi_2)\,, \quad \\
  	C &= a_1 \cos \psi_1 + a_2 \cos (\psi_1 + \psi_2)\,,
  \end{split} &
  \begin{split}
  B &= -a_2 \sin (\psi_1 + \psi_2)\,, \\
  C &= a_2 \cos (\psi_1 + \psi_2)\,,	
  \end{split}
\end{array}\,.
\end{equation*}

\noindent
$T_s$ is the sampling period and $\mathbf{I} \in \mathbb{R}^{10 \times 10}$ the identity matrix.

\begin{equation}
\mathbf{H} = \begin{bmatrix}
  0 & 0 & 0 & 1 & 0 & 0 & 0 & 0 & 1 & 0\\
  0 & 0 & 0 & 1 & 0 & 0 & 1 & 0 & 0 & 1\\
  0 & 0 & 1 & 0 & 0 & 1 & 0 & 0 & 0 & 0\\
\end{bmatrix}\,.
\end{equation}

\begin{equation}
\mathbf{R} = \begin{bmatrix}
  \sigma_1 & 0 & 0\\
  0 & \sigma_2 & 0\\
  0 & 0 & \sigma_3
\end{bmatrix}\,.
\end{equation}

\begin{equation}
\mathbf{Q} = \begin{bmatrix}
  \sigma_d & 0 & 0 & 0 & 0 & 0 & 0 & 0 & 0 & 0\\
  0 & \sigma_d & 0 & 0 & 0 & 0 & 0 & 0 & 0 & 0\\
  0 & 0 & 0 & 0 & 0 & 0 & 0 & 0 & 0 & 0\\
  0 & 0 & 0 & 0 & 0 & 0 & 0 & 0 & 0 & 0\\
  0 & 0 & 0 & 0 & 0 & 0 & 0 & 0 & 0 & 0\\
  0 & 0 & 0 & 0 & 0 & 0 & 0 & 0 & 0 & 0\\
  0 & 0 & 0 & 0 & 0 & 0 & 0 & 0 & 0 & 0\\
  0 & 0 & 0 & 0 & 0 & 0 & 0 & 0 & 0 & 0\\
  0 & 0 & 0 & 0 & 0 & 0 & 0 & 0 & 0 & 0\\
  0 & 0 & 0 & 0 & 0 & 0 & 0 & 0 & 0 & 0\\
\end{bmatrix}\,.
\end{equation}

\section{Implementation}

The filter algorithm was implemented in \textsc{Matlab}$\textsuperscript{\textregistered}$.

\section{Experiments}

\section{Results}

\section{Discussion}

