\chapter{Implementation}
\label{ch:Implementation}

This chapter describes the implementation and applies the fundamentals acquired in the previous the chapters.

\section{Initial Situation}

 (Any algorithms that are already implemented for comparison that I can cite or state and describe how they differ from the one that I will implement??)

\section{Theoretical Design}

According to \citeauthor{bennett_motion_2014} \cite{bennett_motion_2014}, when walking in a straight line, the human leg can be modelled as a two-link planar revolute robot. As depicted in Figure , the revolute joints represent the the hip und knee joints, and the the two links the femur and tibia/fibula, respectively.

\begin{figure}
\centering
\begin{tikzpicture}[auto, thick, node distance=3cm,>=latex']
    \node [draw, very thick, rectangle, minimum height=3.5cm, minimum width=1cm, align=center, rotate around={30:(0,0)}] at (0.6, -0.9) (link1) {};
    \node [draw, very thick, rectangle, minimum height=1cm, minimum width=0.7cm, align=center, rotate around={30:(0, 0)}] at (1, -1) (link1) {};

    
    \node[coordinate] (X) at (2.5,0) {};
    \node[coordinate] (Y) at (0,2.5) {};
    \node[coordinate] (O) at (-0, 0) {};
    
    \draw[->] (O) -- node[name=x] {$x$}(X);
    \draw[->] (O) -- node[name=y] {$y$}(Y);
      
    \end{tikzpicture}

\caption{Acceleration seen by the sensor (b) with and (a) without motion from \cite{bennett_motion_2014}.} \label{fig:acceleration_motion}
\end{figure} 


\section{Implementation}

\section{Experiments}

\section{Results}

\section{Discussion}

