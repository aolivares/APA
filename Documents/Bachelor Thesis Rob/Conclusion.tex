\chapter{Conclusion and Future Work}
\label{ch:Conclusion and Future Work}

\section{Conclusion}

Gait analysis is a useful tool both in clinical practice and biomechanical research. In this work I implemented an extended Kalman filter along with some auxiliary routines that were necessary for testing. The experiments showed that motion-based acceleration correction can improve the angle estimates while walking at different speeds. In order to replace a camera-based motion capture system with low cost wearable MARG sensors, some technical details still need to be improved.

Personally, I have learned a lot in the seven months that I spent in Granada for writing this thesis and completing the internship in advance. Among others, I have come to know many new work methods, due to being exposed to people from different cultures. I gained some experience in scientific research and a deeper understanding of human body motion analysis, in particular gait analysis, as well as how Kalman filtering benefits the accuracy of orientation estimation. I improved my \textsc{Matlab}\textsuperscript{\textregistered} skills and I am now familiar with tools such as GitHub and Pivotal Tracker, which make working in a team much easier and significantly more efficient.  While working at the research centre, I could improve my oral and written English skills.

All in all it was a great experience, professionally as well as personally. I truly and unreservedly recommend such a stay to \emph{every single} university student.

\section{Future Work}

As discussed above, there are a variety of possible improvements regarding the kinematic model, which lead to the following possible future work. One could extend the kinematic model of the leg, in order to take more complex movements of the body and especially the movement of the hip into account. Furthermore, the implemented orientation algorithm needs to be tested with a larger set of data from real subjects, in order to proof its validity and robustness. Especially, whether it delivers accurate orientation estimations without fine tuning the parameters by means of a known reference for every patient and trial needs to be tested. Finally, the acceleration correction could be applied to the accelerometer-based thigh angle estimate as well.

Regarding the GaitWatch system itself, the acceleration correction could be extended to the orientation estimation of the arms and the trunk. Then, the GaitWatch system could replace the camera-based motion capture systems in the future and would thus add a significant value to its medical applications in gait analysis.


