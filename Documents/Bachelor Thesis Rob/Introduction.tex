\chapter{Introduction}
\label{ch:Introduction}

\section{General}


\section{Goals}
\label{data_gathering_protocol}
The goal of the project was to analyse anticipatory postural adjustments prior to step initiation and subsequently build a classifier using MATLAB$\textsuperscript{\textregistered}$, which is fed with data from both a force plate and a magnetic inertial measurement unit (GaitWatch \cite{olivares_vicente_gaitwatch_2013}), in order to distinguish between Parkinson patients and healthy subjects. To gather the data the subject stood in front of the force plate. Then, the GaitWatch and force plate record was started and the subject made a step onto the force plate. After standing upright for a variable time of two to ten seconds the subject left the force plate, made a few steps, turned left, and stopped in front of it again. This sequence was repeated ten times. From now on we will refer to one of these ten gait cycles as simply a cycle.


\section{Motivation}

Advanced \gls{PD} can increasingly diminish the quality of life, due to patients being more dependent on help from others to accomplish daily tasks. New neuroprotective medications are currently being developed and are expected to decelerate or stop the progression of the disease in early stages prior to significant loss of neurons \cite{botzel_motivation_2014}\cite{mancini_isway:_2012}. Thus, a quantitative PD classification, enabling early diagnosis of the disease, could optimise early treatment and could furthermore help to validate new treatment methods. Additionally, an objective evaluation of longterm treatment success was ensured.

\section{GaitWatch}

The GaitWatch device \cite{olivares_vicente_gaitwatch_2013} is a MIMU designed to monitor the motion of patients while attached to the body. It was developed at the Department of Neurology of the Ludwig-Maximilians University in Munich in conjunction with the Department of Signal Theory, Telematics and Communications of the University of Granada. The system is composed of a set of embedded magnetic and inertial sensors wired to a box containing a microcontroller. This microcontroller is in charge of collecting data from the embedded box sensors, as well as from the external measurement units, and storing them on a memory card. The various units are placed at the patient's thighs, shanks, arms and trunk as shown in Figure \ref{fig:GaitWatch_placement}. The components of the three different kinds of subunits are described below:



\begin{itemize}

\item \textsc{Type A} -- thighs and shanks: 

IMU Analog Combo Board with 5 Degrees of Freedom \cite{IMU5} containing an IDG500 biaxial gyroscope (from which only Y axis is actually used) with a measurement range of ±500°/s \cite{IDG500} and a ±3g triaxial accelerometer, ADXL335 \cite{ADXL335}.

\item \textsc{Type B} -- arms:

IDG500 biaxial gyroscope with a measurement range of ±500°/s \cite{IDG500}.

\item \textsc{Type C} -- trunk:

ADXL345 triaxial accelerometer with programmable range \linebreak (±2g/±4g/±8g/±16g) \cite{ADXL345},
IMU3000 triaxial gyroscope with programmable range (±250/±500/±1000/±3000°/s) \cite{IMU3000}, 
Micromag3 \allowbreak triaxial magnetometer with a measurement range of ±11Gauss \cite{MicroMag3}, AL-XAVRB board containing an AVR ATxmega processor \cite{AVRATxmega}.

\end{itemize}


\section{Fundamentals of Inertial Sensors}

In addition to the aforementioned subjective rating scales, there are different devices used to quantify gait and posture to assess them objectively. All of them come with certain pros and cons. The following devices have been used:

\subsection{Accelerometer} Electromyography is a technique for evaluating the electrical activity of skeletal muscles. Successive action potentials generated by muscle cells are measured, by means of needle electrodes inserted into the muscles, and displayed on a cathode-ray oscilloscope. Thus medical abnormalities can be detected. The instrument used to capture the visual recording, termed electromyogram, is called electromyograph \cite{encyclopedia_britannica_electromyography_2014}. Electromyography is constrained to clinical application only, but in return gives indication about the contribution of specific, individual muscles to APAs.

\subsection{Gyroscopes} Force plates quantify the ground reaction force (GRF), that is, the force exerted to the human body by the ground. The GRF is a three-dimensional vector with three orthogonal components. One component along the direction of gravity, one parallel to the ground in the sagittal plane, and one parallel to the ground in the frontal plane. Those are vertical planes that divide the body in left and right halves, and ventral and dorsal sections, respectively. A force plate usually gives an electrical voltage proportional to the force in each of the three directions. Force plates can be characterised according to the following criteria: Sensitivity in Volts per Newton, crosstalk (indication of vertical force if a horizontal force is applied and vice versa), repeatability (similar results under the same load), and time- and temperature drift \cite{griffiths_principles_2006}. Usually force plates are embedded in the ground to place minimum constraints on subjects \cite{mancini_trunk_2011}. Therefore they are limited to clinical application. They have the advantage that they don't need to be calibrated before each use. 

\subsection{Magnetometer} Devices that use a combination of inertial sensors like accelerometers and gyroscopes are referred to as inertial measurement units (IMUs). If they also include magnetic field sensors (magnetometers), they are called magnetic inertial measurement units (MIMUs){Magnetic Inertial Measurement Unit}. With these devices the orientation of the body can be obtained with up to nine degrees of freedom, provided that triaxial accelerometers, gyroscopes and magnetometers are used, respectively \cite{olivares_vicente_signal_2013}.

\begin{itemize}

\item \textsc{Accelerometers} measure the acceleration of an object relative to an inertial frame. Since acceleration cannot be measured directly, the force exerted to a reference mass is obtained and the resultant acceleration is computed according to Newton's second law $ \mathbf{F} = m \cdot \mathbf a $ \cite{encyclopedia_britannica_accelerometer_2014}.

\item \textsc{Gyroscopes} measure angular velocity and are based on the Coriolis Effect. By means of integration of the angular velocity the rotation angle is obtained \cite{olivares_vicente_signal_2013}.

\item \textsc{Magnetometers} measure the strength and the direction of the magnetic field in a point in space, using the relationship between magnetic fields, movement and induced currents \cite{olivares_vicente_signal_2013}.
 
\end{itemize}
MIMUs are portable and relatively inexpensive. They can be easily attached to the body and thus allow non-clinical longterm application. Their drawbacks are complex calibration procedures and drift behaviour over time, depending on intensity and duration of the movement. Hence, in order to maintain a satisfactory degree of precision, periodical recomputation of the calibration parameters is required \cite{olivares_vicente_signal_2013}. Moreover, \citeauthor{mancini_isway:_2012} \cite{mancini_isway:_2012} pointed out the need of data pre-processing and the question of how to generate a clinically understandable presentation of the movement data.


\section{Methodology}

The team in which I was integrated worked using the agile software development methodology. Working software was delivered frequently and was the principal measure of progression. Our self-organising team consisted of three members meeting regularly. Parts of the software were developed doing pair programming. To follow the progress of other team members at any time we used Pivotal Tracker, a tool for agile project management and GitHub, a repository hosting service based on the distributed version control system Git. I used the document markup language \LaTeX{} to write this report.

\section{Document Structure}

Next, in Chapter 2, we will elaborate on the hardware we used and describe the synchronisation process in detail.

