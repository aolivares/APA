\chapter{Introduction}
\label{ch:Introduction}

\section{General}

Orientation estimation assisted by inertial sensors have become an integral part of medical diagnosis and therapy techniques as it provides an objective means of monitoring and assessing human body motion.

\section{Motivation}


\section{Goals}

The goal of this thesis was implementing a new Kalman filter based orientation algorithm proposed by \citeauthor{bennett_motion_2014} in \cite{bennett_motion_2014} to improve the estimation of orientation angles by means of inertial sensors. The filter algorithm should be implemented using MATLAB$\textsuperscript{\textregistered}$ and validated against existing algorithms by comparing their respective error rates. An existing system for human body motion analysis called GaitWatch was available so that no new hardware had to be developed to gather the movement data.

\section{Methodology}

Subsequent to the introductory overview of the subject and the definition of the project objectives, this chapter ends with a description of the state of the art. Chapter 2 outlines the fundamentals of MARG sensors and gives a detailed description of the GaitWatch device. Chapter 3 explains the theoretical basics of digital filters including the Kalman filter as a particular one. Those are necessary for the actual implementation of the filter and the orientation estimation in Chapter 4, in which also the results are presented and discussed. Finally, Chapter 5 concludes and proposes possible future work.

As additional means to communicate with my supervisor and enable him to follow the progress of the project at any time we used Pivotal Tracker, a tool for agile project management and GitHub, a repository hosting service based on the distributed version control system Git. This thesis is written in the document markup language \LaTeX{}.

\section{State of the Art}

MARG sensors can be found in smart phones, fitness trackers, and other wearable devices where they have become a prevalent research environment for estimation and tracking of human body motion \cite{bennett_motion_2014}. They are used in activity monitoring \cite{veltink_detection_96}\cite{najafi_ambulatory_03}\cite{ermes_sports_08}, rehabilitation \cite{giggins_rehabilitation_13}\cite{lupinski_ligament_11}, sports training \cite{bonnet_squat_13}\cite{ermes_sports_08}, and localisation \cite{hoflinger_localization_13}\cite{Bennett_distance_13}. Moreover they have numerous applications in navigation.

There are several research works in the literature dealing with orientation estimation by means of inertial sensors. The state of the art at the commencement of the project is described below. Subsequently, current attempts to revolutionise medical research assisted by inertial sensors are presented.

\subsection{Orientation Estimation}

Given the fact that inertial sensors are used to establish objective body motion parameters that affect medical diagnosis, therapy, and rehabilitation, the need of a high level of accuracy becomes obvious.

\subsection{MARG Sensors in Medical Research}

In March 2015, \citeauthor{Apple_2015} announced an open source framework called ResearchKit \cite{Apple_2015} that, amongst others, takes advantage of the MARG sensors in the iPhone to track movement in patient's daily life. Thus medical researchers obtain robust data with far more regularity than it was possible when patients complete tasks at hospitals or other research facilities in irregularly intervals. Moreover, according to \citeauthor{Apple_2015} ResearchKit simplifies recruiting participants from all over the world which results in a more varied study group that provides a more useful representation of the population.

In August 2014, the Michael J. Fox Foundation for Parkinson’s Research and Intel Corporation \cite{Intel_2013} announced a collaborative research study on objective measurement of Parkinson symptoms. They aim to collect movement data of thousands of patients twenty-four seven at over 300 samples per second by means of unobtrusive wearable devices and store them on a big data analytics platform. The data platform, deployed on a cloud infrastructure, supports an analytics application that processes the data and detects changes in real time. Thus, by detecting anomalies, the progression of the disease can be measured objectively without the need for scientists to focus on the underlying computing technologies. Physicians and researchers are intended to have access to the data as well as be able to submit their own anonymised patient data for analysis. According to \cite{Intel_2013} the correlation of data that quantifies symptoms such as slowness of movement, tremor, and sleep quality with molecular data could advance drug development and provide a deeper insight into the clinical course of Parkinson's disease.



