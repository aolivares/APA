\chapter{Introduction}
\label{ch:Introduction}

\section{General}

Human body motion analysis has become an integral part of medical diagnostic techniques. 


\section{Motivation}


\section{Goals}

The goal of this thesis was implementing a new Kalman filter based orientation algorithm proposed by \citeauthor{bennett_motion_2014} in \cite{bennett_motion_2014} to improve the estimation of orientation angles by means of inertial sensors.



\section{Methodology}

Our Team was composed of three members and worked using the agile software development methodology. Working software was delivered frequently and was the principal measure of progression. To follow the progress of other team members at any time we used Pivotal Tracker, a tool for agile project management and GitHub, a repository hosting service based on the distributed version control system Git. I used the document markup language \LaTeX{} to write this thesis.

\section{Document Structure}

This thesis begins with an introductory chapter where the scope of the concrete task is discussed as well as its goals and ends with the hardware we used. Chapter 2 describes the state of the art. In Chapter 3, we will elaborate on the Fundamentals necessary for the implementation. Chapter 4 presents and discusses the results. Finally, Chapter 5 concludes and proposes possible future work.