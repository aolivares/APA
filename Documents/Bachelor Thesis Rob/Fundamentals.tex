\chapter{Fundamentals}
\label{ch:Fundamentals}

\section{Inertial Sensors}

In addition to the aforementioned subjective rating scales, there are different devices used to quantify gait and posture to assess them objectively. All of them come with certain pros and cons. The following devices have been used:

\subsection{Accelerometer} Electromyography is a technique for evaluating the electrical activity of skeletal muscles. Successive action potentials generated by muscle cells are measured, by means of needle electrodes inserted into the muscles, and displayed on a cathode-ray oscilloscope. Thus medical abnormalities can be detected. The instrument used to capture the visual recording, termed electromyogram, is called electromyograph \cite{encyclopedia_britannica_electromyography_2014}. Electromyography is constrained to clinical application only, but in return gives indication about the contribution of specific, individual muscles to APAs.

\subsection{Gyroscopes} Force plates quantify the ground reaction force (GRF), that is, the force exerted to the human body by the ground. The GRF is a three-dimensional vector with three orthogonal components. One component along the direction of gravity, one parallel to the ground in the sagittal plane, and one parallel to the ground in the frontal plane. Those are vertical planes that divide the body in left and right halves, and ventral and dorsal sections, respectively. A force plate usually gives an electrical voltage proportional to the force in each of the three directions. Force plates can be characterised according to the following criteria: Sensitivity in Volts per Newton, crosstalk (indication of vertical force if a horizontal force is applied and vice versa), repeatability (similar results under the same load), and time- and temperature drift \cite{griffiths_principles_2006}. Usually force plates are embedded in the ground to place minimum constraints on subjects \cite{mancini_trunk_2011}. Therefore they are limited to clinical application. They have the advantage that they don't need to be calibrated before each use. 

\subsection{Magnetometer} Devices that use a combination of inertial sensors like accelerometers and gyroscopes are referred to as inertial measurement units (IMUs). If they also include magnetic field sensors (magnetometers), they are called magnetic inertial measurement units (MIMUs){Magnetic Inertial Measurement Unit}. With these devices the orientation of the body can be obtained with up to nine degrees of freedom, provided that triaxial accelerometers, gyroscopes and magnetometers are used, respectively \cite{olivares_vicente_signal_2013}.

\section{Kalman Filter}