\chapter{Abstract}

The analysis of human gait can assist the diagnosis of diseases, and can help to assess treatment success in rehabilitation. In order to estimate the orientation of the human body, inertial sensors have become increasingly important, as they mitigate the drawbacks of camera-based motion capture systems. Using a kinematic model of the leg, the acceleration due to motion was subtracted from the readings of a biaxial accelerometer on the shank, in order to improve the accelerometer-based pitch angle estimate. This improved estimate was fused with measurements of another biaxial accelerometer on the thigh, and two uniaxial gyroscopes on the thigh and shank in an extended Kalman filter. The filter algorithm was applied to movement data from a real subject, which walked on a treadmill at different speeds. The filter improved the overall pitch angle estimate by an average root-mean-square error that was $\unit[28.52]{\%}$ smaller than the one of an existing classical Kalman filter. It was concluded that motion-based acceleration correction can benefit the accuracy of the pitch angle estimates, but further testing of the robustness of the filter algorithm with a larger set of data is proposed. Additionally, a more complete mathematical model could further improve the angle estimates.
